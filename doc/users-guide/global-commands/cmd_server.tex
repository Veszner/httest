% =========================================
% COMMAND: SERVER
% =========================================

\newpage
\section{SERVER}
\label{cmd:SERVER}

\paragraph{Syntax:}
\subparagraph{}
\texttt{SERVER [<SSL>:]<addr\char`\_port> [<number of concurrent servers>] [ -> <host>:<por> }

\paragraph{Purpose:}
\subparagraph{}
Begin of a body which starts one or multiple threads running a test server for responding  
to HTTP requests in predefined ways.

The number of threads started depends on the value of the argument for 
the number of concurrent servers. If none is given, it defaults to 1 and 
only a single thread will be started. Otherwise, the same \texttt{SERVER} body will 
be executed in the specified number of parallel threads.

The \texttt{SERVER} must be closed using the \texttt{END} command.

If the server should speak SSL put one of the possible <SSL> tags in front of the <addr\char`\_port>. Possible <SSL> tags are SSL, SSL2, SSL3, TLS1. Depending of the linked openssl you also can set TLS1.1 or TLS1.2.

With -> <host>:<port> you can define a remote host, where this server must run. The remote host must start the following command 'htremote -p <port> -e "httest -Ss"'. The server will be serialized and send over to remote httest which then executes this server.

% Please see chapter TODO-REF for details about how to correctly set up SSL connections.
