% =========================================
% COMMAND: MODULE
% =========================================

\newpage
\section{MODULE}
\label{cmd:MODULE}

\paragraph{Syntax:}
\subparagraph{}
\texttt{MODULE <name>}

\paragraph{Purpose:}
\subparagraph{}
A module represents a collection of \texttt{BLOCKs}. It adds a namespace 
element that can help organize re-usable blocks by adding the module 
name as a prefix for each block.

So, every block contained in a module must be invoked by adding the 
module name as a prefix.

\begin{usplisting}
    # Begin a module "utility"
    MODULE utility

    BLOCK one
    ...
    END
\end{usplisting}

Invoke the block "one" from this example then like that:

\begin{usplisting}
    _CALL utility:one
\end{usplisting}
