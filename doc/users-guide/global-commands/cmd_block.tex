% =========================================
% COMMAND: BLOCK
% =========================================

\newpage
\section{BLOCK}
\label{cmd:BLOCK}

\paragraph{Syntax:}
\subparagraph{}
\texttt{BLOCK <name>}

\paragraph{Purpose:}
\subparagraph{}
Begins a body with scripting code that can be invoked from anywhere 
else, using the \texttt{CALL} or \texttt{\char`\_CALL} command.

The \texttt{BLOCK} must be closed using the \texttt{END} command.

\paragraph{Error Handler}
\subparagraph{}
A special, reserved name for a \texttt{BLOCK} is \texttt{ON\_ERROR}. It 
designates the block to be invoked if an unexpected error occurs during 
a test. The block can then handle the error, output some information etc. 
Example:

\begin{usplisting}
    BLOCK ON_ERROR
    _DEBUG
    _DEBUG !!!!!!!!!!!!!!!!!!!!!!!!!!!!!!!!!!!!!!!!!!!!!!!
    _DEBUG An error occurred, please fix it and try again!
    _DEBUG !!!!!!!!!!!!!!!!!!!!!!!!!!!!!!!!!!!!!!!!!!!!!!!
    _DEBUG
    _EXIT FAILED
    END
\end{usplisting}
