\chapter{htt Commands}
\label{chap:commands}

\textit{Commands} are the actual instructions that make up the test script, such as 
sending a request to a HTTP server, evaluating the response etc. There are two 
different types of commands:

\begin{itemize}
\item \textit{Global} commands always begin with a capital letter (e.g. "\texttt{INCLUDE}") and 
      can be placed in the "root" of the script, outside of any body (block).
\item \textit{Local} commands always begin with an underscore (e.g. "\texttt{\_GET}") and 
      can only be placed within a \textit{body}.
\end{itemize}

Some commands take arguments to perform their operation. Usually, the syntax is

\begin{usplisting}
    COMMAND [argument] [argument] [argument] ...
\end{usplisting}

There is a special exception, the double-underscore local command (see section 
"Local Commands" for details) which does not take a blank character between the 
command and its argument.

% ---------------------------- Global commands -----------------------------

\newpage
\section{Global Commands}

Global commands are used to perform initialization work, clean-up steps, 
start and end bodies etc.

Global commands can only be used within the "root" part of the script, 
i.e. outside of a body. It is not possible to use a global command within 
a \texttt{CLIENT}, \texttt{SERVER}, \texttt{DAEMON} or \texttt{BLOCK} body. 

Note that some commands can are available in two forms, both as global 
and local command.

The following global commands perform an operation by themselves:

\begin{itemize}
\item \texttt{SET} - Sets a global variable.
\item \texttt{INCLUDE} - Includes content from a file.
\item \texttt{GO} - 
\item \texttt{EXEC} - 
\item \texttt{TIMEOUT} - 
\item \texttt{AUTO\_CLOSE} - 
\item \texttt{FILE} - 
\end{itemize}

While the global commands listed below are used to start or end 
a body (a block of script code):

\begin{itemize}
\item \texttt{CLIENT} - Starts a \texttt{CLIENT} body.
\item \texttt{SERVER} - Starts a \texttt{SERVER} body.
\item \texttt{DAEMON} - Starts a \texttt{DAEMON} body.
\item \texttt{END} - Ends the current body.
\end{itemize}

All these commands are explained in more detail in TODO-REF.

% ---------------------------- Local commands -----------------------------

\newpage
\section{Local Commands}

Local commands always begin with an underscore ("\_") character. There are a 
large number of local commands, and they are explained in more detail in TODO-REF.

\subsection*{Double Underscore Command}

Please note that the double-underscore is also a local command! But contrary to 
all other commands, with the double-underscore there's no blank character between 
the command and it's parameters.

The reason for this somewhat peculiar design is that this command is used to 
insert a line of data into the output stream of a HTTP request. And a line always 
ends with two invisible characters for the line-break and line-feed. Now, in some 
cases it is necessary to calculate the exact length of a line in order to know the 
correct length of the HTTP request body; in that case, the length of a line 
displayed by a text editor can be used directly as the length of the actual line 
of HTTP data for body length calculations.

% ============================================================================
% ---------------------- List of GLOBAL commands -----------------------------
% ============================================================================

\newpage
\section{List Of Global Commands}


% ---------------------- Command: AUTO_CLOSE -----------------------------
\subsection{AUTO\_CLOSE}

\paragraph{Syntax:}
\subparagraph{}
\texttt{AUTO\char`\_CLOSE on|off}

\paragraph{Purpose:}
\subparagraph{}
Closes the current connection automatically.


% ---------------------- Command: BLOCK -----------------------------
\subsection{BLOCK}

\paragraph{Syntax:}
\subparagraph{}
\texttt{BLOCK <name>}

\paragraph{Purpose:}
\subparagraph{}
Begins a body with scripting code that can be invoked from anywhere 
else, using the \texttt{CALL} or \texttt{\char`\_CALL} command.

The \texttt{BLOCK} must be closed using the \texttt{END} command.


% ---------------------- Command: CLIENT -----------------------------
\subsection{CLIENT}

\paragraph{Syntax:}
\subparagraph{}
\texttt{CLIENT [<number of concurrent clients>]}

\paragraph{Purpose:}
\subparagraph{}
Begin of a body which starts one or multiple threads running a test client for sending 
HTTP requests to a server, and evaluating the response.

The number of threads started depends on the value of the argument for 
the number of concurrent clients. If none is given, it defaults to 1 and 
only a single thread will be started. Otherwise, the same \texttt{CLIENT} body will 
be executed in the specified number of parallel threads.

The \texttt{CLIENT} must be closed using the \texttt{END} command.


% ---------------------- Command: DAEMON -----------------------------
\subsection{DAEMON}

\paragraph{Syntax:}
\subparagraph{}
\texttt{DAEMON}

\paragraph{Purpose:}
\subparagraph{}
Begins a body with scripting code that is executed in a separate 
thread, and can be used to perform tasks like monitoring the 
tests, enforcing timeouts etc.

The \texttt{DAEMON} must be closed using the \texttt{END} command.


% ---------------------- Command: END -----------------------------
\subsection{END}

\paragraph{Syntax:}
\subparagraph{}
\texttt{END}

\paragraph{Purpose:}
\subparagraph{}
Ends the current body (either a \texttt{CLIENT}, \texttt{SERVER}, \texttt{DAEMON} 
or \texttt{BLOCK} body). 


% ---------------------- Command: EXEC -----------------------------
\subsection{EXEC}

\paragraph{Syntax:}
\subparagraph{}
\texttt{EXEC <shell command>}

\paragraph{Purpose:}
\subparagraph{}
Executes the given command, e.g. an external shell script. Note: The 
execution will not join any of the CLIENT or SERVER threads (TODO: WAS 
GENAU HEISST DAS?)


% ---------------------- Command: FILE -----------------------------
\subsection{FILE}

\paragraph{Syntax:}
\subparagraph{}
\texttt{FILE <name>}

\paragraph{Purpose:}
\subparagraph{}
Creates a temporary file with the given name. TODO: WHAT FOR? EMPTY? HOW TO USE?


% ---------------------- Command: GO -----------------------------
\subsection{GO}

\paragraph{Syntax:}
\subparagraph{}
\texttt{GO}

\paragraph{Purpose:}
\subparagraph{}
Starts all \texttt{CLIENT}, \texttt{SERVER} and \texttt{DAEMON} bodies that 
have been defined up to this point. 

TODO: WANN IST DAS NOETIG? BEI EINEM EINFACHEN CLIENT JA NICHT.


% ---------------------- Command: INCLUDE -----------------------------
\subsection{INCLUDE}

\paragraph{Syntax:}
\subparagraph{}
\texttt{INCLUDE <include file>}

\paragraph{Purpose:}
\subparagraph{}
Includes the given file into the current script code. The file name parameter 
is processed relative to the working directory of the current process, if it 
is not an absolute path.


% ---------------------- Command: SERVER -----------------------------
\subsection{SERVER}

\paragraph{Syntax:}
\subparagraph{}
\texttt{SERVER [<SSL>:]<addr\char`\_port> [<number of concurrent servers>]}

\paragraph{Purpose:}
\subparagraph{}
Begin of a body which starts one or multiple threads running a test server for responding  
to HTTP requests in predefined ways.

The number of threads started depends on the value of the argument for 
the number of concurrent servers. If none is given, it defaults to 1 and 
only a single thread will be started. Otherwise, the same \texttt{SERVER} body will 
be executed in the specified number of parallel threads.

The \texttt{SERVER} must be closed using the \texttt{END} command.

Please see chapter TODO-REF for details about how to correctly set up SSL connections.


% ---------------------- Command: TIMEOUT -----------------------------
\subsection{TIMEOUT}

\paragraph{Syntax:}
\subparagraph{}
\texttt{TIMEOUT <timeout in ms>}

\paragraph{Purpose:}
\subparagraph{}
Sets the global TCP socket connection timeout.


% ============================================================================
% ---------------------- List of LOCAL commands -----------------------------
% ============================================================================

\newpage
\section{List Of Local Commands}

\subsection{\char`\_\char`\_ (Double Underscore)}

\paragraph{Syntax:}
\subparagraph{}
\texttt{\_\_<data>}

\paragraph{Purpose:}
\subparagraph{}
Bla bla