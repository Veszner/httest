\chapter{htt Commands}
\label{chap:commands}

\textit{Commands} are the actual instructions that make up the test script, such as 
sending a request to a HTTP server, evaluating the response etc. There are two 
different types of commands:

\begin{itemize}
\item \textit{Global} commands always begin with a capital letter (e.g. "\texttt{INCLUDE}") and 
      can be placed in the "root" of the script, outside of any \textit{body} (such as a block).
\item \textit{Local} commands always begin with an underscore (e.g. "\texttt{\_GET}") and 
      can only be placed within a \textit{body}.
\end{itemize}

Some commands take arguments to perform their operation. Usually, the syntax is

\begin{usplisting}
    COMMAND [argument] [argument] [argument] ...
\end{usplisting}

There is a special exception, the double-underscore local command (see section 
"Local Commands" for details) which does not take a blank character between the 
command and its argument.

% ---------------------------- Global commands -----------------------------

\newpage
\section{Global Commands}
\label{chap:globalCommands}

Global commands are used to perform initialization work, clean-up steps, 
start and end bodies etc.

Global commands can only be used within the "root" part of the script, 
i.e. outside of a body. It is not possible to use a global command within 
a \texttt{CLIENT}, \texttt{SERVER}, \texttt{DAEMON} or \texttt{BLOCK} body. 

Note that some commands are available in two forms, both as global 
and local command.

The following global commands perform an operation by themselves:

\begin{itemize}
\item \texttt{AUTO\_CLOSE} - 
\item \texttt{EXEC} - 
\item \texttt{FILE} - 
\item \texttt{GLOBAL} - Defines global variables.
\item \texttt{GO} - 
\item \texttt{INCLUDE} - Includes content from a file.
\item \texttt{PROCESS} - Runs the test in multiple processes.
\item \texttt{SET} - Sets a global variable.
\item \texttt{TIMEOUT} - 
\end{itemize}

While the global commands listed below are used to start or end 
a body (a block of script code):

\begin{itemize}
\item \texttt{CLIENT} - Starts a \texttt{CLIENT} body.
\item \texttt{SERVER} - Starts a \texttt{SERVER} body.
\item \texttt{DAEMON} - Starts a \texttt{DAEMON} body.
\item \texttt{END} - Ends the current body.
\end{itemize}

All these commands are explained in more detail in \ref{chap:globalCommandsList}.

% ---------------------------- Local commands -----------------------------

\newpage
\section{Local Commands}
\label{chap:localCommands}

Local commands always begin with an underscore ("\_") character. There are a 
large number of local commands, and they are explained in more detail in \ref{chap:localCommandsList}.

\subsection{Double Underscore Command} 

Please note that the double-underscore is also a local command! But contrary to 
all other commands, with the double-underscore there's no blank character between 
the command and it's parameters.

The reason for this somewhat peculiar design is that this command is used to 
insert a line of data into the output stream of a HTTP request. And a line always 
ends with two invisible characters for the line-break and line-feed. Now, in some 
cases it is necessary to calculate the exact length of a line in order to know the 
correct length of the HTTP request body; in that case, the length of a line 
displayed by a text editor can be used directly as the length of the actual line 
of HTTP data for body length calculations.
