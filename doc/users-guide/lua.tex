\section{Extentions with Lua}
If you have enabled Lua support in your httest you can use the full Lua
power in htt. Lua fits perfekt into httest, because Lua also do have the feature
of multiple return values like httest. 

\subsection{Lua blocks}
With a special block \texttt{BLOCK:LUA} you can embedd Lua into httest. Paired
with the possibility to define parameter and return value signature you are
able to define new fancy httest commands.

A simple Lua example:

\begin{usplisting}
    BLOCK:LUA my_lua_func text param val : res res2
      print(text.." "..param.." = "..val)
      return "hello", "world"
    END
      
    CLIENT
      my_lua_func "That is my text" "key" "abcdefg" ret ret2
      _DEBUG $ret $ret2
    END
\end{usplisting}

A simple but powerfull extention of \texttt{\_WAIT} makes it possible to hand over a
received buffer to a Lua block. Additional we need a way to hand over the buffer
unresolved. 

A simple example will explain it:

\begin{usplisting}
    BLOCK:LUA my_lua_func buffer
      print("Received buffer:")
      print(buffer)
    END
      
    CLIENT
      _REQ www.wikipedia.com 80
      __GET / HTTP/1.1
      __Host: www.wikipedia.com
      __
      _WAIT buf
      
      my_lua_func VAR(buf)
    END
\end{usplisting}

There are two special things which makes it possible to hand over a received 
buffer to a Lua block. The first thing is, \texttt{\_WAIT} can store the buffer to a
variable, if one is specified. The other special thing is \texttt{VAR(buf)}
which do hand over the received buffer unresolved. Of course you can do it
the old way with \texttt{\$buf}. But if the result is not a single word you have
to quote it and if there are quotes in the received buffer it, it wont work at all,
because the resolving is allwas done before execution. That's why we use
the trick with \texttt{VAR(buf)}. 

\subsection{Special Lua extentions}
To make the lua integration more powerfull a set of special httest commands are
accessable from Lua.

