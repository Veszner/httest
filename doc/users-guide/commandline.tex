\chapter{htt Commandline Interface}
\label{chap:commandline}

\section{Files}

The htt tool is available for multiple platforms. On Unix platforms, the htt 
provides two binaries:

\begin{itemize}
 \item \texttt{httest}: The actual HTTP testing tool.
 \item \texttt{htntlm}: A tool required to perform an NTLM authentication.
\end{itemize}

On Windows, a number of other files (.exe and .dll) are required to run the htt.

\section{Syntax}

The htt binary uses the following commandline syntax:

% ---------------------- SAMPLE BOX ----- BEGIN ------------
\begin{usplisting}
    %> httest [OPTIONS] scripts
\end{usplisting}
% ---------------------- SAMPLE BOX ----- END ------------

Where \textit{scripts} is just one or multiple ".htt"-files, for example:

% ---------------------- SAMPLE BOX ----- BEGIN ------------
\begin{usplisting}
    %> httest -Ts testone.htt testtwo.htt
\end{usplisting}
% ---------------------- SAMPLE BOX ----- END ------------

File names must be specified either relative to the current working 
directory, or with an absolute path.
\newpage 
\section{Options}

A list of all options can always be retrieved by using the "--help" argument:

% ---------------------- SAMPLE BOX ----- BEGIN ------------
\begin{usplisting}
    %> httest -Ts testone.htt testtwo.htt

    httest is a script based tool for testing and benchmarking web 
    applications, web servers, proxy servers and web browsers. httest 
    can emulate clients and servers in the same test script, very 
    useful for testing proxys.

    Usage: httest [OPTIONS] scripts

    Options:
      -V --version         Print version number and exit
      -h --help            Display usage information (this message)
      -n --suppress        do no print start and OK|FAILED
      -s --silent          silent mode
      -i --info            log level info
      -e --error           log level error
      -w --warn            log level warn
      -d --debug           log level debug
      -L --list-commands   List all available script commands
      -C --help-command    Print help for specific command
      -T --timestamp       Time stamp on every run
      -S --shell           Shell mode
   
    Examples:
    httest script.htt
   
    httest -Ts script.htt
   
    Report bugs to http://sourceforge.net/projects/htt   
\end{usplisting}
% ---------------------- SAMPLE BOX ----- END ------------

