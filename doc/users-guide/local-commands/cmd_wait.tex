% =========================================
% COMMAND: _WAIT
% =========================================

\newpage
\section{\_WAIT}
\label{cmd:_WAIT}

\paragraph{Syntax:}
\subparagraph{}
\texttt{\_WAIT [<amount of bytes> | <variable>]}

\section{Purpose:}
Waits for data on a connection and reads it. Without this 
command, the script would just continue after sending out 
data, without reading the response.

\subparagraph{}
\texttt{\_EXPECT} and \texttt{\_MATCH} commands will be 
applied to the incoming data at this point. This is why 
they must always preceed the \texttt{\_WAIT} command. 

\subparagraph{}
Optionally, it's also possible to just read a given amount 
of bytes.

\subparagraph{}
If you pass a variable, the read data is stored in the given
variable. If the variable hasn't been declared yet, then it
is created as a local variable.
