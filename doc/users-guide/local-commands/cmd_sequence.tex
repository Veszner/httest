% =========================================
% COMMAND: _SEQUENCE
% =========================================

\newpage
\section{\_SEQUENCE}
\label{cmd:_SEQUENCE}

\paragraph{Syntax:}
\subparagraph{}
\texttt{\_SEQUENCE <variable-sequence>}

\paragraph{Purpose:}
\subparagraph{}
Defines a sequence of \texttt{\_MATCH} or \texttt{\_GREP} variables which must 
appear in the given order in the response.

\subparagraph{}
With \texttt{\_MATCH} or \texttt{\_GREP}, it is possible to define certain 
values that are expected in a response, one at a time. But the order of 
those values within the response is not determined by those commands. 
In the following example, the \texttt{\_MATCH} commands make sure that 
the response contains three strings "one", "two" and "three". And the 
\texttt{\_SEQUENCE} command then makes sure that they are in exact 
this order within the response:

\begin{usplisting}
    CLIENT
    _REQ localhost 8080
    __GET /index.html HTTP/1.1
    __Host: localhost
    __
    _MATCH body "(one)" first
    _MATCH body "(two)" second
    _MATCH body "(three)" third
    _SEQUENCE first second third
    _WAIT
    END
\end{usplisting}
