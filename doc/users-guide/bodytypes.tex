\chapter{Body Types}
\label{chap:bodytypes}

The following global body types exist:

\begin{itemize}
\item \texttt{CLIENT} - The main HTTP test client logic.
 
\item \texttt{SERVER} - Allows to start a HTTP server which will react and respond as defined in the 
 given block. This allows to simulate HTTP back-end servers (like some application server)
 that may not be available in the testing environment.
 
\item \texttt{BLOCK} - A block is a named section of the script with a defined begin and ending, not 
 unlike functions or methods in common scripting or programming languages. 

  \begin{itemize}
    \item \texttt{BLOCK FINALLY} - A special block. The FINALLY block is always executed at the end of
    the test, and can be used for clean-up work.
    \item \texttt{BLOCK ON\_ERROR} - Another special block. A global error handler that will be executed 
    when an error occurs in a test script. While some errors can be handled within a regular body (e.g. during 
    the evaluation of a HTTP response), other errors (especially technical ones) sometimes can only be 
    handled in this block.
  \end{itemize}
  
\item \texttt{BLOCK:LUA} - This defines a special block for embedding Lua into httest. Lua support must
be enabled on compile time (\texttt{./configure --enable-lua-module}). 
 
\item \texttt{DAEMON} - Other than a \texttt{SERVER}, a \texttt{DAEMON} does not implement any kind of listener. It is just 
a separate thread that can be used to monitor or control the execution of the test, e.g. implement timeout 
functionality to interrupt a hanging test etc.

\end{itemize}


The following local body types exist:

\begin{itemize}
\item \texttt{\_BPS} - Limits the number of bytes sent during a period of time.
\item \texttt{\_ERROR} - Ensures that a certain expected error happened within the body.
\item \texttt{\_FOR} - Allows to define code with is executed for a given number of elements.
\item \texttt{\_IF} - Allows to define code with is executed only if a given condition is true.
\item \texttt{\_LOOP} - Allows to execute the given block of code a certain number of times.
\item \texttt{\_RPS} - Limits the number of requests sent during a period of time.
\end{itemize}


\newpage 
\section{Bodies And Threads}

\subsection{Parallel Execution}
Note that all body blocks are executed in separate threads! So, if a htt 
script contains more than one \texttt{CLIENT} body, each one will be started in a 
separate thread, allowing to simulate more distinct, separate HTTP clients in 
one test case.

\subsection{Execution Order}
While multiple bodies are executed in parallel threads in general, \texttt{SERVER}
and \texttt{DAEMON} bodies are always started before \texttt{CLIENT} bodies.

\subsection{Lifespan}
If all \texttt{CLIENT} and \texttt{SERVER} bodies end, then
the test ends, and the htt process exits. The still running \texttt{DAEMON} threads are
automatically terminated.

\subsection{Synchronization}
With the \texttt{\_LOCK} and \texttt{\_UNLOCK} commands it is possible to
synchronize those threads. 


% --------------------------- Section "Client" ------------------------------

\newpage
\section{CLIENT}
\label{chap:client}

A \texttt{CLIENT} body is the part of a htt script, where the actual HTTP requests 
are generated to perform some kind of test. Usually, there is just one single 
\texttt{CLIENT} body in a htt script, but there can be multiple ones; in that case, 
they are executed in parallel threads.

A htt test ends, when all \texttt{CLIENT} and \texttt{SERVER} bodies complete their work.

A simple \texttt{CLIENT} example:

% ---------------------- SAMPLE BOX ----- BEGIN ------------
\begin{usplisting}
    CLIENT
    _REQ www.wikipedia.org 80
    __GET / HTTP/1.1
    __Host: www.wikipedia.org
    __
    _EXPECT . "200 OK"
    _EXPECT . "Wikipedia"
    _WAIT
    END
\end{usplisting}
% ---------------------- SAMPLE BOX ----- END ------------


% --------------------------- Section "Server" ------------------------------

\newpage
\section{SERVER}
\label{chap:server}

A \texttt{SERVER} body can start a HTTP listener and react to incoming HTTP 
requests with predefined responses. This allows to simulate application 
backend servers that may not be available in a test environment. 

There can be multiple \texttt{SERVER} bodies; in that case, 
they are executed in parallel threads.

A htt test ends, when all \texttt{CLIENT} and \texttt{SERVER} bodies complete their work.

A simple \texttt{SERVER} example:

% ---------------------- SAMPLE BOX ----- BEGIN ------------
\begin{usplisting}
    # Simple HTTP listener which just responds with
    # a HTTP 200 return code, and an empty body.
    SERVER
    _RES
    _WAIT
    __HTTP/1.1 200 OK
    __Content-Length: AUTO
    __Content-Type: text/html
    __

    END
\end{usplisting}
% ---------------------- SAMPLE BOX ----- END ------------


% --------------------------- Section "Daemon" ------------------------------

\newpage
\section{DAEMON}
\label{chap:daemon}

A \texttt{DAEMON} body does not start a listener, and also not necessarily 
send any HTTP request. It's purpose is to perform monitoring tasks during 
a test. For instance, if a \texttt{CLIENT} body sends a HTTP request that 
may hang for a long time due to slow response time of the server, the 
\texttt{DAEMON} may enforce an exit of the htt test process after a certain 
time, essentially implementing a timeout function.

This is only one possibility. A \texttt{DAEMON} can also monitor log files 
written during the test, execute external binaries etc.

There can be multiple \texttt{DAEMON} bodies; in that case, 
they are executed in parallel threads.

A htt test does not end when the \texttt{DAEMON} bodies complete their work, 
only when the \texttt{CLIENT} bodies have finished.

A simple \texttt{DAEMON} example:

% ---------------------- SAMPLE BOX ----- BEGIN ------------
\begin{usplisting}
    # Timeout daemon thread, which exits the 
    # test process with a failure return code,
    # if it takes longer than 30 seconds.
    DAEMON
    _SLEEP 30000
    _DEBUG Test duration too long
    _EXIT FAILED
    END
\end{usplisting}
% ---------------------- SAMPLE BOX ----- END ------------


% --------------------------- Section "Block" ------------------------------

\newpage
\section{BLOCK}
\label{chap:block}

A \texttt{BLOCK} body is not executed by itself like \texttt{CLIENT} or 
\texttt{SERVER} bodies; it just serves as a holder for a fragment of 
re-usable script code, similar to a function or method in common 
programming languages.

Every \texttt{BLOCK} must have a name, in order to distinguish them from 
each other, and to be able to be invoked from another body.

Just like high-level programming language functions or methods, 
a \texttt{BLOCK} can be parameterized, in a way 
very similar to shell scripts. The name of the \texttt{BLOCK}, mentioned 
in the previous paragraph, is just the first parameter (and it's 
mandatory). Within the \texttt{BLOCK} body, all parameter values are 
available as variables with simple numeric names, like in a shell script:

\begin{itemize}
  \item \texttt{\$0} - The name of the \texttt{BLOCK}
  \item \texttt{\$1} - The first parameter after the name
  \item \texttt{\$2} - The second parameter after the name
  \item \texttt{\$3} - The third parameter, and so on
\end{itemize}

A simple \texttt{BLOCK} example, without parameters:

% ---------------------- SAMPLE BOX ----- BEGIN ------------
\begin{usplisting}
    BLOCK SENDREQ
    _REQ www.acme.com 80
    __GET / HTTP/1.1
    __Host: www.acme.com
    __
    _WAIT
    END
\end{usplisting}
% ---------------------- SAMPLE BOX ----- END ------------

The following example shows a \texttt{BLOCK} with some parameters:

% ---------------------- SAMPLE BOX ----- BEGIN ------------
\begin{usplisting}
    # Block takes 3 parameters:
    # 1: hostname, like www.acme.com
    # 2: port
    # 3: requested URI path
    BLOCK SENDREQ
    _REQ $1 $2
    __GET $3 HTTP/1.1
    __Host: $1
    __
    _WAIT
    END
\end{usplisting}
% ---------------------- SAMPLE BOX ----- END ------------

New it is possible to specify input and output parameters in the block
definition. This input and output parameters are only visible with in the block.
The parameters are space separated. The input parameters are separated from 
the output parameters with a "\texttt{:}". The return values are mapped
to the callers variables.

A simple \texttt{BLOCK} example with defined paramters:

% ---------------------- SAMPLE BOX ----- BEGIN ------------
\begin{usplisting}
    BLOCK SENDREQ host url : code text
    _REQ $host 80 
    __GET $url HTTP/1.1
    __Host: $host
    __
    _MATCH headers "HTTP/1.1 ([0-9]+) (.*)" code text
    _WAIT
    END

    CLIENT
      SENDREQ www.wikipedia.com / my_code my_text
      _DEBUG $my_code $my_text
    END
\end{usplisting}
% ---------------------- SAMPLE BOX ----- END ------------

