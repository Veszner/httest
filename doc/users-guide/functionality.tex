\chapter{htt Functionality Overview}
\label{chap:functionality}

\begin{itemize}

 \item The htt provides a large variety of HTTP-related functionality, useful for 
implementing all kinds of HTTP-based tests:

 \item Advanced HTTP protocol handling, including fine-grained timeout handling, 
request and response validation

 \item Simulating clients and servers, including startup and shutdown of server 
daemons. This allows to create mock-ups of back-end systems in more complex 
test situations, for example when the tested application needs to interact
with a 3rd-party back-end system which is not available in the testing 
environment.

 \item Execution of external commandline tools, using their output as request 
or response data, or for validation purposes.

 \item Copying stream data (e.g. from a response) and re-using it in variables.

\end{itemize}

\chapter{htt Functionality Changes}
\label{chap:Changes}

There are various changes to older httest 2.0 versions.

\begin{itemize}

\item Variable names are now case-sensitiv.

\item Variable names ending with round braces will be interpreted as inline
function calls. To avoid function calling embrace the variables in curl braces.
For example you have the following line

\begin{usplisting}
__Any text $Foo(.*) bar
\end{usplisting}

Foo is called as a command and the return value will replace this
inline command. If this is not what you expected you have to change this to

\begin{usplisting}
__Any text ${Foo}(.*) bar
\end{usplisting}


\end{itemize}
